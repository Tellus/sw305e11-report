\chapter{Theory}

\section{Usability}
When developing software and other products that require user interaction, one must be aware of the usability of the product and how to evaluate it.
Usability is a quality attribute, which is fundamental to a successful product and is defined by 5 quality components:
\begin{itemize}
	\item Learning: How easy is it for users to accomplish tasks during first encounter of the design. 
	\item Efficiency: After learning how the product works, how fast can they perform various tasks.
	\item Memorability: After a period of not using a product, how easily can they obtain efficiency again.
	\item Errors: How many errors do users make and how easily can they recover from the errors.
	\item Satisfaction: How pleasant is it to use the product.
\end{itemize}

To explain the importance of usability and how it is fundamental to a successful product, one must understand that no matter how a product functions, as long as user interaction is required, it must comply with the needs of the user and help to complete the tasks at hand.
As an example one can look at a website. If a website is difficult to use, if users get lost trying to browse the website or the information stored on the website is hard to read and/or figure out, people tend to look for alternatives. This results in loss of customers and therefore is bad for the company that designs, hosts and administrates the website.
If a product on the other hand is designed for a special purpose where no alternatives are to be found, like a system to administrate a hospital. Usability is so important that errors could lead to serious harm of patients. Reasons can be efficiency or errors, where time spent figuring out the system means delay in service and errors can result in system stalling or misinformation.

There are methods for studying and improving usability but user testing has proven the most useful and will therefore be the project�s focus when evaluating usability of the product.\cite{usability}