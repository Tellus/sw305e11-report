

This project was based at some android applications made in a bachelor multiproject in 2011. The developed system was called GIRAF. The initializing problem from the original projectproposal, for this project, was to make an PC tool, used for administration of the applications developed for the GIRAF system. During the project the group has been cooperating with Birken, which is a institution for autistic children, and after the first meeting with the reprecentiv Kirstine Niss-Henriksen, the focus of the project was changes from making an administration tool, to making a digital version of the contact book used by Birken to communicate between the institution ant the parents of the children.
%The initializing problem of this rapport was to make an PC-administrations tool that would be able to administer Giraf system on a mobile device, that was made in a bachelor multi-project in 2011. But after interviewing the kindergarten teacher Kristine Niss-Henriksen from Birken the rapport focus on making a  digitized version of the children's contact book that simplify the communication between the kindergarten teachers and parents. 
This rapport describes the development of Giraf administration system with main focus on the contact book. This rapport includes the usage of models, prototypes, the client-server architecture, and the implementation of the model-view-controller design. Furthermore to evaluate the system both unit testing and usability testing used. 