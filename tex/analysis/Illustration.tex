\section{Illustration of the system}
A \emph{rich picture} is created to help one understand the interaction within a specific technology and/or product\cite{OOAD}. The drawing in figure \vref{fig:Rig billede} is an overview of how the users (parents and kindergarten teachers) and the child would interact with the mobile device. Under the assumption that the multi-bachelor project \emph{GIRAF} has been implemented in everyday life, the intention is to show how it could be improved with an administrative application on the computer.

The mobile device is primarily used by the child as a communication tool by using the application DigiPECS or aSchedule, but games can be installed too, to entertain the child. As a communication tool the child can ask for the user for help and the user can use the tool to communicate with the child or instruct the child to, e.g wash their hands before dinner.
Further more the users also need to use the mobile device when they want to add or edit something on the device, for example when scheduling the next day's activities in the aSchedule application.
 
The current issue with GIRAF-A-11 is that both the administrator and child need physical access the same mobile device. Some children will consider the mobile device theirs and therefore will not hand it over to the user or do so with much frustration. To resolve this problem without changing the mobile device's primary function, a computer application could be developed, so the user could administer the mobile device from a computer via some form of a wireless connection. All this is illustrated in figure \vref{fig:Rig billede}. 

\begin{figure}[!ht]
	\centering
		\includegraphics[width=1.00\textwidth]{img/Rig_billede2.jpg}
	\caption{The child uses the tablet to communicate with the parent or kindergarten teacher, so the child and adult can understand each other. The parent or kindergarten teacher uses a computer to administrate the tablet}
	\label{fig:Rig billede}
\end{figure}
\newpage

A meeting with the group's contact from Birken was conducted prior to the remainder of the design process. This resulted in a significant change in focus
%Before we started designing, we decided to conduct our first interview with our contact. This resulted in a different focus point.
    