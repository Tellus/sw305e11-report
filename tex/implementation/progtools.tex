\section{Designing tools}
For development of the system, four tools were used, the markup language \textit{HTML} along with the style sheet language \textit{CSS}. And then \textit{PHP} and \textit{JavaScript}, both scripting languages. These are some of the most common web-design tools and have easy-access documentation. All four are currently supported by the most common browsers, including but not limited to \textit{Google Chrome}, \textit{Mozilla Firefox} and \textit{Internet Explorer}.
The reason for so many tools is the difference in functionality, \textit{HTML} provides the browser with information about how to build a website, along with \textit{CSS} that defines the structure, layout and general styling. \textit{PHP} and \textit{JavaScript} have provide functions and data-manipulation that markup languages do not provide.
\cite{html}\cite{css}\cite{php}\cite{javascript}

\subsection{HTML \& CSS}
\textit{HTML} is an acronym for HyperText Markup Language. With elements, the building blocks of \textit{HTML} one can create a website exclusively with text, layout, images and more, although \textit{CSS} is recommended for layout and styling. \textit{CSS} is an acronym for Cascading Style Sheets and is designed primarily to separate the website contents from the presentation.
What \textit{CSS} provides with the separation is structure and readability for the developers since the code itself is not visible to the users and \textit{CSS} provides no extra functionality.\cite{html}\cite{css}

\subsection{PHP}
\textit{PHP} is an acronym for PHP: Hypertext Preprocessor and is a server-side scripting language. As the name inclines, it is preprocessed, on the website server, to automatically write \textit{HTML} code. In other words \textit{PHP} is used to generate results displayed in \textit{HTML}. While \textit{PHP} can run recursive functions, generate code and transport data, it is static in view, all the work is handled by the server hosting the website, which then only displays results. To develop dynamic web content, \textit{JavaScript} or equivalent is needed, which brings us to the next section.\cite{php}

\subsection{JavaScript}
\textit{JavaScript} not to be confused with \textit{Java}, is a scripting language that provides dynamic web content. That is able through client-side execution of web related scripts. The browser gets instructions from the website about how to execute the proposed content, where it is the user's choice of whether to run these instructions.
Since most browsers support dynamic web content through \textit{JavaScript} execution, many developers run scripts on their websites to increase the visual aspect of the website and to make dynamic checking of forms and spelling for example.\cite{javascript}