\chapter{Introduction}

The purpose of this project is to create a user-friendly web-application that makes the user able to create entries in an web-based contact book which should replace the current china book version and make it easier for the kindergarten teachers and guardians to communicate with each other. Furthermore, this project will try to establish a foundation, such that the functionalities created in this project can be added to the current GIRAF system. 

The project is created as addendum to other GIRAF projects, casting light on how these additions can be made to the already implemented software, moreover how it would improve the working environment for kindergarten teachers working with autistic children. 
Specifically, one institution contributed throughout the project, Birken, which is a kindergarten for children with a need for special care and a alternative learning environment. Birken is situated in Vodskov in the Northern Denmark region.

The primary contact from Birken, Kristine Niss-Henriksen, has been included in this project because of her knowledge about children with autism and experience with today's technology. Furthermore, she contributed with design ideas for the front-end of the software and in testing and providing feedback on prototypes, which helped us in creating the final product.

The project was originally based on creating an administration tool for the GIRAF tablets, making it possible to manage and push application settings to registered devices. However, from the interviews made with Kristine we learned that this functionality was not a pressing issue; adding the aforementioned contact book functionality had a higher priority.  

Some words are defined in the list below for the reader's convenience
\begin{itemize}
\item{Guardians = Parents, or persons which has custody over the child}
\item{Kindergarten teacher = In this context, a person working with autistic children}
\item{The GIRAF system is the result of a multi-project made in spring 2011 which is a android based operating system for a mobile device that creates a safe environment for children with disabilities.}
\end{itemize}