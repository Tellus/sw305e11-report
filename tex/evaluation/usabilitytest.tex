\section{Usability testing process}
To receive feed back on our prototype, we decided to conduct a usability test. Our testing subject would be an actual user, in this case the kindergarten teacher Kristine Niss-Henriksen.
\subsection{Planning}
% write out testing times (the third meeting) and note what we wanted to test (which features?) and which attributes of those features we'd like to determine the usability of (for example, ease of use, if enough of the features are supported, etc). 

In our usability test we wanted feedback on a number of areas; one being the general layout, such as the double menu, creating a combination of the two axes. We wanted to test the intuition of the menu, as it is one the core parts of the prototype. 
 
Furthermore we wanted to receive feedback on our contact book, which has been a focus point in our prototype. The prototype should provide a simple overview for the user. To test this, we instructed our subject to carry out simple tasks to see how the user reacted.


\subsection{Third meeting \& testing}
% already in process, this was very informel remember to note that.

During the interview(see appendix \vref{third_interview}) we showed the prototype to our subject. She was first presented with a login screen, where the subject was instructed to create a new user. After a login was created the subject proceeded to login and was directed to the main window. We gave a brief introduction to the layout and proceeded to guide our subject through further tasks, such as creating a new post in the contact book, adding images to a post and reading existing posts. 


\subsection{Evaluation}
% what did we learn from the meeting. 
Our first attempt to run the prototype failed as our subject used an older browser on her computer, which is an unavoidable problem and should be fixed.

When our subject was first introduced to the login screen, she created a new user and logged in without trouble. However we noted that she expected the mouse to automatically focus on the input field, after she typed her information incorrectly. Furthermore it was not possible to use the Enter-key to submit user credentials. This should be easy to patch.

When our subject was presented with the main window, she immediately turned to the menu for the next step. Although the menu was buggy, we managed to pull through. As our subject was asked to create a new post in the contact book she mistook the headline submit box for the body text, which means; the layout should be different. 
Furthermore we received feedback on how the contact book should work, i.e. how and when a post is marked as ``new`` or ``unread``. We presented an idea in which a new post would lose its status as ``new`` if simply one colleague opened the post. This could improve efficiency as the individual could read a post on behalf of the kindergarten. However the kindergarten teachers at the kindergarten would read every mail regardless of content, the idea of sharing ``New post`` with colleagues was abandoned. 
Additionally she emphasized the idea of having a print function, which would print posts from a child's contact book, as this often plays a major role in the child's life. This function should be easy to access.


