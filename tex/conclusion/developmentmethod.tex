\section{Development methodology}

The use of prototypes was essential to ensure that more than one perspective were taken in when designing. With the user's involvement and testing on the protoypes, time was saved when narrowing down to a system definition and choosing a focus point. Protoypes were created on blackboard, paper and in code. The blackboard prototypes helped the group visualise how changes would affect the interface and helped involving everyone, within the group, in designing. The code implemented prototypes helped the group realise how difficult some functions were to implement.

To develop the product, the group split into three smaller groups, each including two individuals; that partnered up to write code and help each other while developing.
The original idea was to rotate these small groups between programming tasks; to introduce everyone to all aspects of the project and to prevent people from getting stuck.
Although the idea was good, the difference in experience with developing and lack of communication resulted in groups sticking with their original tasks.
In development, some parts and tasks took much longer time than predicted, which resulted in tasks beeing pushed on to others.

The group analysed the problem domain, designed the product and then focused solely on development, which was driven in contrast with the agile method, but without purging unusable code.

