\section{Guardians}
Describe the relationship between children with autism and their guardians.
\subsection{PACT}
A PACT analysis was performed on the current application of the contact book. The single parameters are described in their respective headers.
\subsubsection*{People - persona}
Christine is in her thirties and is a fully educated pedagogue. Beyond this she has taken several degrees and courses focusing on the needs of children with mental handicaps, all resulting in her now working at the kindergarten for special-needs children, Birken.

She takes care of a select few children of those attached to the kindergarten, an activity that requires almost constant supervision while they are present. Christine and her colleagues yearn for solutions that can ease and improve their daily work, particularly within the domains of communication with the children and their parents. Very notably, the time required to add a picture to an entry in the child's contact book is so severe that she cannot do it very often, a hindrance that bothers her. She would very much like to be able to attach an image to the book every day to improve the dialogue between child and parent.
\subsubsection*{Activities}
\begin{itemize}
	\item Each day an entry is written in the contact book of every child, detailing the highlights of the day. It may be a single sentence or a paragraph with a picture of the child engaging in an activity. This is done by the guardians at the kindergarten, but entries may be commented or entered anew by parents with any important thoughts.
	\item If a picture is to be added, first the picture is taken with a digital camera, imported into a desktop computer, then cropped, resized or rotated as necessary. Finally, the image is printed onto paper, cut out and glued into the contact book.
	\item Child and parent will sit down with the book, using the image in the book (if present) as a starting point for dialogue about the day, continuing through available pectograms.
\end{itemize}
\subsubsection*{Contexts}
There are three contexts based on the three activities, two of them sufficiently similar to be written as one.
\begin{itemize}
	\item When compiling an entry in the contact book, the guardian or parent will sit by themselves, the child occupied either with their own devices or the supervision of someone else. 
	\item When discussing the entry in the book, however, child and parent are together.
\end{itemize}
\subsubsection*{Technology}
Currently, two hardware technologies are in play: A digital camera, and a personal computer.
An image manipulation suite is used to edit the photographs. Possibly, this is a Windows pre-installed application like MS Paint.
\subsubsection*{Conclusion}
Even at a glance, several possible improvements can be suggested that make use of IT. Central are the notions of digitising the contact book and easing the creating of images by using a mobile phone or tablet with image manipulation software capable of the actions usually necessary.