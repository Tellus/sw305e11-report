\section{Communication with an autistic child}
The parents and kindergarten teachers use Picture Exchange Communication System or \emph{PECS}, to communicate with an autistic child and this system has been successfully used for over 10 years in the USA\cite{centerAutism}. The communication tool consists of pictures or pictograms that represent something the child should do or wants to do. This can be a drawing of an apple, signaling the child which fruit to eat. 

This form of communication is used in kindergartens for children with special needs e.g. children with autism. Mostly these pictograms are printed from a computer program called \emph{Boardmaker}, which has many different pictograms that can be edited to make it easier for the child to understand\cite{centerAutism}. This could be when the child needs to put on a blue t-shirt, then the adult gives the child a pictogram of a blue t-shirt and the child understands. It can be confusing for the child if the actual t-shirt is one color and the one on the pictogram is another.

The PECS can also be used as a schedule for the children. In the kindergarten Birken pictograms are used as a daily schedule, where kindergarten teachers put the pictograms along a column on the wall and each row represents a single activity.
The child is taught to take the pictogram of the finished activity and put it away in a box. Later, the pictogram can be put on next day's schedule, without confusing the child about whether the activity is finished or not.
%Later in the child's development the pictogram would be put next to the schedule, but the child would still understand that this activity is finished.
