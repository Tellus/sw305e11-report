\section*{Third interview with Birken}
\label{third_interview}

0.00
The prototype i started up.
Kristine is asked to create a new user.

1.00
The prototype is not working, so Johannes tells about the project.

2.12
Changes from Birken's computer, to a laptop brought along, because \texit{Internet Explorer} cannot run the prototype.

9.30
The \textit{Enter} button cannot be used to submit changes. Kristine was expecting more focus in the login-screen.

11.40
Kristine is presented for an overview of the children.
Some bugs in the menu.

13.00
Kristine is asked to make a new post in the contactbook. She confuses the boxes with each other.

14.00
Kristine is asked to add a picture and she chooses one from the computer.

15.00
The picture is successfully uploaded.

16.00
The picture is added along with text, but meanwhile shows pictures form an other earlier post.

16.40
It is discussed whether is should be possible to include pictures to an answer, and Birken is positive about that possibility.

17.30
Kristine tries to write an answer to the uploaded post

18.30
Kristine thinks, the answer function works well with date, sender and message. She would like to upload a picture and then answer.

19.30
Kristine is told about our ideas about the overall project/solution, with a newsside, and general messages.

20.40
Kristine likes the functionality about uploading pictures.
It is discussed, whether there should be picture text or not.

21.20
Should be accessible for the child \todo{should this not be explained more?}

21.40
It is discussed whether is should be possible to edit pictures. Kristine focus at cutting pictures.

22.40
Kristine asks for a printfunction. She would like to have a general and easy accessible printfunction for all posts.

23.30
Kristine mentions the datalaw, because the pictures must not be stored at the computer for more than 3 months, and therefore a printfunction would be good, because the child can the keep the pictures.

24.00
A discussion about the law about personal data is taking place, and the internet is searched for information.

28.30
The searching stops, but the discussion continues.

30.40
Problem about the serves located in England, where the law may be different.

33.00
Questions about the budget of Birken is asked to Kristine, but she do not know about it, because it is the manager who keeps track of it. Economy about the possibility of having Birken's own server is discussed and Kristine is informed about different possibilities in that area.

34.40
Respites and our distinguish between pedagogues and guardians are discussed.

37.00
Kristine is open to the possibility of not only autistic children using the system.

38.10
The remaining part of the project period is discussed.

40.30
Kristine talks about the good things about the design
\begin{itemize}
\item{Easy login}
\item{Easy to access the children}
\item{Easy to see and read the news}
\item{Easy to upload pictures}
\item{Easy for people with low computer experience}
\item{Easy to understand the functions of the buttons}
\end{itemize}

41.15
Kristine is focusing at the print-function for the contactbook

42.00
Kristine is not expecting more from the visual design, than what already is made.

44.00
Kristine is told about the principle of a usability-test, and focus at the possible problem with having an older browser.
The interview/test is summed up.

46.30
The design of the system is further explained, including the principle of the menu with child/application.

48.00
A discussion about the word 'settings', which can be too insensitive, and Kristine agrees. She suggest 'contact informations'.

49.30
Kristine suggest some sort of gallery, where is should be possible to gather all pictures to an album of photos to the children. The possibility of the GIRAFAdmin to be opened by the children is brought up.

51.50
A mysterious bug is detected by Kristine

53.30
Kristine prefer if 'new' disappears when the entry has been read, which she compares to e-mail and 'Outlook'. She also mentions the possibility of marking a entry, so it would be easy to locate later.

54.30
It is discussed, whether is should be visibly to see if other people have read an entry. Kristine prefers to have a distinction between these. Although all entries have to be read by all pedagogues and the conclusion becomes, not to have this relation between entries and users.

56.30
The function which makes it possible to see if an answer to an entry has appeared is discussed, and Kristine is positive about it.

58.40
The layout of 'new entry' is discussed once again, and Kristine comes up with some improvements.

61.40
The idea of having profilepictures is told to Kristine and she seems positive about the idea.

62.40
Different types of grouping, which could be practical because the department in Vodskov has two groups of children, is discussed. Kristine is positive about the idea.

65.30
Kristine gets access to the system, so she can show the webpage to her colleagues.

68.30
Kristine is referring to a sms-system, which is used by the employees. 