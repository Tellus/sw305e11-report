% under chapter reflection som 5. section
\section{Expanding the problem domain}
During this project, the problem domain focus has been an institution for children with autism, however the GIRAF-A-11 could be used by both children and adults with different disabilities and therefore the GIRAF-W-11 could be used within other institutions. The contact book with its current functionality, could be used in schools and be adapted to adults with a disabilities, as a diary.     

The GIRAF concept could be expanded further to support individuals, that have problems using a mobile device. An example, the elderly who are uncomfortable using a modern mobile device because of the extensive functionalities and its small buttons. To accommodate these users, a safe environment, like the GIRAF system, with different levels of restrictions could be utilized.

The GIRAF concept could also be used for children without disabilities, where GIRAF Administration and the use of a contact book, could enhance communications between parents and kindergarten or school teachers. Furthermore the GIRAF-W-11 could include an application that forwards the child's homework to the parents, containing solutions, so that the parents are better able to help with the homework. 

If the framework reaches a stable release (Android synchronization client, synchronization server and the possibility to manage groups) any organization or group in need of a closed application distribution and management form could use parts of the framework to some extent - taking this a step further could be, extending the GIRAF platform to include and manage any type of application.