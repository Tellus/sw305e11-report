\chapter{System definition}

\section{BATOFF}
This is a BATOFF analysis of the online journal for parents and pedagogues to communicate in the everyday life. The journal should be available from the tablet such that a pedagogue can snap a picture, write a post and upload it for the kindergarten teacher or guardian to read. BATOFF is a tool to define a product.

\subsection{Betingelser - Conditions}
There are several requirements for the journal to work:
\begin{itemize}
	\item There has to be a connection between the mobile devices, such as the tablets, and the platform. 
	\item It should be easily accessed by people with little IT-experience, as we have very little knowledge about the users, especially the parents.
\end{itemize}

\subsection{Anvendelsesomr�de - Scope}
The scope for the journal concerns the pedagogues, parents and guardians and in some regards the autistic children. Furthermore there will be some administrators maintaining the system:
\begin{itemize}
	\item The pedagogue is a main user of the journal as he or she mostly fills out the journal. The pedagogue should have access to the different children's journal and can upload and edit pictures to add to the journal.
	\item The kindergarten teacher or guardian should only be able to read the journal of their own child, but can edit and add pictures.
	\item The child can read and show their parent etc. the pictures in the journal, but are unable to edit the journal.
\end{itemize}

\subsection{Teknologi - Technology}
There are a couple of technological requirements for the journal to work:
\begin{itemize}
	\item	The platform should be supported by both mobile devices as well as stationary devices.
	\item A database to store information, such as user credentials.
  \item A specially designed framework to handle the journal application.
	\end{itemize}

\subsection{Objekter - Objects}
The main objects for the journal consist of these:
\begin{itemize}
	\item	Mobile devices i.e. the tablet.
	\item	The autistic child as they can read the journal.
	\item	Pictures that pedagogues can take for the journal. 
	\item	The parents/guardians and the pedagogues as the journal serves as a communication between the two. 
	\item	A day, as the journal is most likely edited from day to day and does not matter from i.e. week to week.
\end{itemize}

\subsection{Funktioner - Functions}
The journal itself has a number of functions:
\begin{itemize}
	\item	Write a post of the day for the parents or guardians to read.
	\item	It should be available for both pedagogues and parents to edit and delete posts. 
	\item	Take, upload and edit pictures for the journal. Furthermore pictures should be able to be attached to the post of the day.
	\item	The picture editor should include crop, resize and rotate.
	\item	It should be possible to upload posts from both the mobile device and the stationary.
	\item	(It should be possible for users to see other authorized users.)
\end{itemize}

\subsection{Filosofi - Philosophy} 
The main idea behind this journal is to:
\begin{itemize}
	\item	Aid to the completion of writing and adding pictures to posts.
	\item	Give an administrative tool to manage the journal.
	\item	Improve the communication between pedagogue and parent and the communication between the parent and the child.
\end{itemize}

\section{System definition}
This piece of software is primarily a communication enhancing tool for the guardians and pedagogue for the daily use and secondly a conversation starter with the child. It should be able to send a story/text and pictures back and forwards between the guardians and pedagogue. This should be done by a platform which is supported both on mobile devices and stationary devices.