% under chapter reflection som 5. section
\section{Expanding the problem domain}


During this project we have based the product upon the needs of a kindergarten that has children with autism, however, this GIRAF system could be used by both children and adults with different disabilities and therefore the GIRAF administration system could be used by other institution, that works with these individuals. 

Furthermore, the product could be expanded towards the elderly. The product is meant to support individuals having problems using normal computers.  

The GIRAF system could also be used for children without any disability, where GIRAF Administration and the contact book then could be used between parent and kindergarten or school teacher. -Lisbeth

For schools: an application could be develops such that the child's homework where entered every day, and solutions(just to the parent) such that the parent are able to help the child with his/hers homework.(meant for children up to 10-11 years) 

If / when the framework reaches a stable release (Android synchronization client, synchronization server and the possibility to manage groups) any organization or group in need of a closed application distribution and management form could use code to some extent - taking this a step further could be to extending the GIRAF platform to include and manage any type of application. -Christoffer

    