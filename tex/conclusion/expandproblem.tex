% under chapter reflection som 5. section
\section{Expanding the problem domain}


During this project we have based the product upon the needs of a kindergarten that has children with autism, however, this product could be used by kindergarten and schools for children with different disabilities. -Lisbeth

evt. Adults with disabilities? -Lisbeth
%make it simple and say individuals with different disabilites, to cover more than just children. Birgir

Furthermore, the product could be expanded towards the elderly. The product is meant to support individuals having problems using normal computers. -Toke 

The GIRAF system and GIRAF Administration could be used for children without any disability, where the contact book then could be used between parent and kindergarten or school teacher. -Lisbeth
%Administration administration?  fixed:lisbeth
%I'm not that much into the other application developed so far, but is it all the apps that can by used for children with no disabilities?

For schools: an application could be develops such that the child's homework where entered every day, and solutions(just to the parent) such that the parent are able to help the child with his/hers homework.(meant for children up to 10-11 years) -Lisbeth


%Christoffer
If / when the framework reaches a stable release (Android synchronization client, synchronization server and the possibility to manage groups) any organization or group in need of a closed application distribution and management form could use code to some extent - taking this a step further could be to extending the GIRAF platform to include and manage any type of application. -Christoffer

    