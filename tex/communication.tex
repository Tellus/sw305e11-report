\section{Communication with the autistic child}
The guardian and kindergarten teachers use Picture Exchange Communication System (PECS) to communicate with the autistic child and this system have successfully been used for over 10 years in USA\cite{centerAutism}. The communication tool is pictures or pictograms that would represent something the child are to do or something the child wants to do. This could be a drawing of an apple and then the child knows that it is time to eat an apple. 

This form of communication is used by the kindergartens for special needs children e.g. children with autism. Mostly this pictogram is printed from the computer program Boardmaker that has many different pictograms which can be edited such that it would be easier for the child to understand. This could be a when the child need to put on a blue t-shirt then the adult give the child a pictogram of a blue t-shirt and the child knows what it mean. If the t-shirt would have been red but if in the pictogram the t-shirt is blue then it would be confusing for the child. 

The PECS could also being used as a schedule for the child. In the kindergarten Birken they use the pictogram as a daily schedule, where the kindergarten teacher put the pictograms on the wall so one row would represent the day's activities for one child. Then the kindergarten teacher teach the child that when an activity is finish then the child should take the pictogram form the schedule and put it away in a box. Later in the child's development the pictogram would be put next to the schedule, but the child would still understand that this activity is finished. 