\section{Model-View-Controller}
The model-view-controller (MVC) is an architectural pattern, that separates an application into three parts and has a well defined area of responsibility. The model is responsible for all the calculation and data handling, that the controller needs; in fact all the work the application can do is done in the model. The view is the interface that the user interacts with and it is a representation of the model. The controller gets input from the user and is then able to get data or change data from the model, but the controller can also change the view in response to the user input\cite{vmc}. MVC can be used as a framework e.g. to create web applications.

The first iteration of \emph{GIRAF-W-11} followed a more generic three-tier architecture with model, business logic and view. The initial assumption was the view as single interaction point for the user and the remaining architecture as downwards strict. This approach created a very rigid and unextendable structure; that, if extended, created unnecessary complications. There was no central point of access control, views were requested directly by clients and adding functionality (the contact book) proved more a question of hardcoding over flexible adaptation.

Inspired by the web-based MVC frameworks Ruby on Rails (programmed in \emph{Ruby}) and CodeIgniter (programmed in \emph{PHP}), the business logic and view layers were completely remodelled to fit an MVC architecture introducing a central gateway (the \emph{index.php} file) for all user interaction. Furthermore, user input was delegated to the controller. The benefit is that controllers now determine which view (or views) should be used with a particular request and the same request can result in two very different displays (for example, one fit for small-screen mobile devices and one for wide-screen PC monitors) while business logic remains completely unchanged.

The remainder of this section will discuss the technical design aspects of the three layers individually. For details on final implementations, see chapter \vref{chap_mvc_start}.

% The begin of the design process we used a The Three-tier Architecture with model, business logic and view, however, this created problem in the implementation. Then we decided to use a Model-View-Controller framework because that was more efficient and in the end easier to implement. This decision would also make it easier for future developers to add more functionality, e.g. an aSchudule administrator application. Another advantage of the MVC framework is that the view can be replaced so if Giraf administration application should be displayed on a mobile device then different view could be used.

\subsection{Model}
\label{design_tech_model}
As the foundation on which the entire framework rests, the model needs to be simple to use, yet offering powerful flexibility. The Active Record pattern is an architectural pattern related to relational databases. A database table is represented by a class and a single row of the table as an object of that class. Knowing how a table is usually structured (columns with specific types of data as well as indices determining uniqueness and searchability), the basic class for a record can be defined universally.
In the simplest terms, a single record in a table consists of the following:
\begin{itemize}
    \item One or more fields of data. Each field contains a specific type of data, and may or may not be a null value.
    \item One or more indices connected to the fields. At least one field, the primary key, must be defined.
    \item Zero or more associations with other record types.
\end{itemize}
Some, if not all, of the structure can be analysed from the source database. The SQL standard specifies the INFORMATION\_SCHEMA table containing information about all schemas in the database. Furthermore, several implementations support a DESCRIBE statement simplifying the analysis even more. In essence, a record type's structure can be uniquely determined using only its source table.
The final result is a single class that requires minimal effort to inherit in specialisations (only changing the source table), leaving later designers to focus solely on the properties that make a particular specialisation unique.

\subsection{Controller}
As the next immediate layer after the model, the controller is best described before the view (in contrast to the literal name of the architecture).
% Discuss how designing a triad of controllers - primary, secondary and supporting - makes sense from the point of view of security, control and extendability. Clearly defined roles give clearly defined types of output.
From MVC certain aspects of the controller are clear. Examples of functions for the controller can be seen in figure \ref{fig_controller_cb}. It must receive user input and affect changes in the model depending on said input. It should also instruct the view layer to display the correct/relevant output to the user.
Beyond this there are no restrictions on how the controller should relate (in structure) to the model or view. For example, there are no demands that a specific part of the model reside in a 1:1 relationship with a particular controller. Indeed, a strength of the design is the controller's mandate to create an amalgam from the model that fits the situation for which it was created.
From a technological standpoint, the controller - as the point for user input and origination of output - could be considered the only exposed element of the server that GIRAFAdmin resides on. Birken has earlier stressed the need for absolute security in the implementation (particularly in regards to personal information and images), and creating a virtual bottleneck in resource availability should result in a degree of control sufficient for these purposes.
Finally, the specific controllers necessary for the current specification can be discerned. The controllers should be designed around certain contexts, and given the function list created during the early design process works as an indicator. As an example, the functions defined for the contact book have several similarities to \emph{CRUD} (for more information see \vref{CRUD}), the basic functions of database storage.

\begin{figure}[ht]
	\begin{itemize}
		\item Create new message
		\item Create reply
		\item Attach image
		\item Show a message
		\item List messages
	\end{itemize}
	\caption{\label{fig_controller_cb}Function list for contact book controller.}
\end{figure}

\subsection{View}
MVC specifies that the view needs to reflect changes in the model, but does not explicitly define when the view is updated.
The controller determines which view will fit a request and instructs the view in what to display, typically by passing it a subset of the model pertaining to the request. This suggests that views be designed according to two axis: the client type (text-based, mobile device, PC, and such) and the content type (more specifically, tailored to the controller most likely to invoke the view). While the controller determines which view to display, the question of how (or when) state changes in the model are shown remain undetermined in the MVC model. The two base options are using either the observer pattern, updating the view by request of the model whenever states change, or the polling pattern, updating the view only when this is explicitly requested. Choosing an approach is more a question of prudence in accordance with the available technology than considering the resulting system complexity. In order to make a decision it is necessary to delve, briefly, into the target domain of web software. HTTP is, by definition, a system of action-reaction. A client will never be affected to perform actions from remote sources, effectively limiting input to go outwards (client to server) and output inwards (server to client).

For the sake of argument, consider the contact book, and the user requesting a list of relevant messages. The client requests a list from the server and the appropriate view is invoked, rendered and returned. If, after the page is loaded, someone creates a new relevant message for the user, the server has no direct way of iniating a change in the view.
It is possible to create the effect of an observer pattern by using timed polling client-side. Implementing JavaScript logic into the final view would instruct the client to poll the server for changes at regular intervals, updating the view as necessary. However, this is a feature unique to the view and does not create the observer connection sometimes detailed between model and view. Instead, it stresses the necessity for views to be largely self-contained where possible.

The conclusion of this is that views need to be built within the contexts of purpose and target platform with the understanding that they must implement an active polling component if state changes need to be reflected without direct user interaction.
% Debate that in a web-server context, designing views to work as polling devices and not observers is a proper consideration.
