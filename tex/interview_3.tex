\section{Third interview with Birken}
\label{third_interview}

0.00
The prototype is started up.
Kristine is asked to create a new user.

1.00
The prototype does not work, so Johannes explains the project concept.

2.12
Switch from Birken's computer, to a laptop brought along, because \emph{Internet Explorer} cannot run the prototype.

9.30
The \emph{Enter} button cannot be used to submit changes. Kristine expected more focus in the login-screen (that the page would automatically select the first form when one starts to write) .

11.40
An overview of the children is presented to Kristine.
Some bugs found in the menu.

13.00
Kristine is asked to create a new post in the contactbook. She mistakes the text field for a headline.

14.00
Kristine is asked to add a picture and she chooses one from the computer.

15.00
The picture uploades successfully.

16.00
The picture is added along with text, but pictures from previous posts are shown aswell.

16.40
It is discussed whether it should be possible to include pictures in an answer, Birken is positive about that possibility.

17.30
Kristine tries to reply to the uploaded post

18.30
Kristine thinks, the answer function works well with date, sender and message. She wants to try to upload a picture and then answer.

19.30
Kristine hears our ideas about the overall project/solution, with a news-site, and general messages.

20.40
Kristine likes the functionality around uploading pictures.
It is discussed, whether there should be picture text or not.

21.20
Should be accessible for the child \todo{should this not be explained more?}

21.40
It is discussed whether it should be possible to edit pictures. Kristine stresses the need for image cutting.

22.30
Kristine asks for a print function. She would like to have an easily accessible print function for all posts.

23.30
Kristine mentions the datalaw, because pictures may not be stored at the computer for more than 3 months. Therefore a printfunction would be good, allowing the child to keep its pictures.

24.00
A discussion about the law about personal data takes place, where a search is conducted on the internet.

28.30
The searching stops, but the discussion continues.

30.40
An issue, including servers located in England, is presented, where the law may be different.

33.00
Questions about Birken's budget, are asked. Kristine has no knowledge about that because it is the manager who keeps track of it. Economy about the possibility of having Birken's own server is discussed and Kristine is informed about different possibilities in that area.

34.40
How the group distinguishes between pedagogues and guardians, is discussed.

37.00
Kristine is open to the possibility of all children using the system.

38.10
The remaining part of the project period is discussed.

40.30
Kristine talks about the good things about the design
\begin{itemize}
\item{Easy login}
\item{Easy to access the children}
\item{Easy to see and read the news}
\item{Easy to upload pictures}
\item{Easy for people with low computer experience}
\item{Easy to understand the functions of the buttons}
\end{itemize}

41.15
Kristine stresses the need for a print-function for the contactbook

42.00
Kristine does not expect more from the visual design, than what already is implemented.

44.00
Kristine is told about the principle of a usability-test, she points out the problem with having an older browser.
The interview/test is summed up.

46.30
The design of the system is further explained, including the principle of the menu with child/application.

48.00
A discussion about the word 'settings', which can be too insensitive, and Kristine agrees. She suggests 'contact information'.

49.30
Kristine suggest some sort of gallery, where it should be possible to gather all pictures into an album for the children. The possibility of opening GIRAFAdmin to the children is brought up.

51.50
A mysterious bug in the menu, is detected by Kristine

53.30
Kristine prefers that the 'new' label disappears when the entry has been read, which she compares to e-mail and 'Outlook'. She also mentions the possibility of marking a entry, so it would be easy to locate later.

54.30
It is discussed, how visible it should be for individuals, that other people have read an entry. Kristine prefers to have a distinction between these. Although all entries have to be read by all pedagogues. In conclusion it is decided not to have this relation between entries and users.

56.30
The function which makes it possible to see if an answer to an entry has appeared is discussed, and Kristine is positive about it.

58.40
The layout of 'new entry' is discussed once again, and Kristine comes up with some improvements.

61.40
The idea of having profile pictures is expressed to Kristine and she seems positive about the idea.

62.40
Different types of grouping is discussed, which could be practical because the department in Vodskov has two groups of children. Kristine is positive about the idea.

65.30
Kristine gets access to the system, so she can show the webpage to her colleagues.

68.30
Kristine refers to an sms-system, which is used by an employee of Birken. 