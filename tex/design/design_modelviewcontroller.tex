\section{Model-View-Controller}
The model-view-controller (MVC) is an architectural pattern which separates an application into three parts which has a well defined area of responsibility. The model is responsible for all the calculation and data handling that the controller needs in fact all the work the application can do is done in the model. The view is the interface that the user interacts with and it is a representation of the model. The controller gets input from the user and is then able to get data or change data from the model, but the controller can also change the view in response to the user input\cite{vmc}. MVC can be used as a framework e.g. to web applications.

The begin of the design process we used a The Three-tier Architecture with model, business logic and view, however, this created problem in the implementation. Then we decided to use a Model-View-Controller framework because that was more efficient and in the end easier to implement. This decision would also make it easier for future developers to add more functionality, e.g. an aSchudule administrator application. Another advantage of the MVC framework is that the view can be replaced so if Giraf administration application should be displayed on a mobile device then different view could be used.