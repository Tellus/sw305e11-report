%er under chapter future work som 1. section
\section{Further development}

\subsection{Android}
We did not make an application for the child such that he/she would be able to see the images and image text from the contactbook. - Lisbeth
%Christoffer
However, the designs for the protocol, client and server have been documented; implementing the features in protocol milestone 1.0 would make it a lot easier to develop an Android viewing application for the contact book, since it contains the basic functionality needed to synchronize and fetch data. In general; the next step would be to implement those features would be the next step. 
%Evt. skal Milestone 1.0 indsættes her - kommer når jeg lige har fået den på papir. Christoffer.  
%FUU too many steps ! (read rest of last sentence in Android). Birgir
\subsection{myAccount and News}
\label{myAccount}
%Maybe it should say somthing like: MyAccount and the News homepage was designed but not complitily implented - Ren� rettet:lisbeth
MyAccount and the News homepage was designed but not completely implemented. The model for both my account and News is done, but both controller and view is missing.

As in our design the myAccount would contain personal information and the user would be able connect a child to him/herself and connect the child with an device. Furthermore all the user's connected children would be able to be made into groups by the user.

In news a parent should be able to receive general news from the Kindergarten, while the kindergarten teachers also can add news.  

\subsection{Application administration}
We did not get to making an application administration system to aSchedule or DigiPECS. These administrations system would need the android synchronization such that the mobile devices' will be updated. 

The aSchedule administration would be like a normal calendar that also uses PECS to illustrate the child's schedule for a day. 

DigiPECS administration would be a program where the user could use to make and edit PECS, and it should then have access to an public archive of PECS (evt. opensource where the user also can add PECS) and a private archive for private pictures and PECS. 

\subsubsection*{Personalize}
This section should describe the "Settings" application, in which the user can edit information about the child and add/remove eventual handicaps. (It should also not be called "Settings" as it is an insensitive name.


\subsection{Site Admin}

\subsection{Delevoper guide}

\subsection{Remote installation}

