\chapter{System definition}
In this chapter a FACTORS analysis and a system definition, listing the demands for the digitised contact book, is outlined.

\section{FACTORS}
This is a FACTORS analysis of the contact book, a communication tool for parents and kindergarten teachers. The contact book should be available from the mobile device, so that a kindergarten teacher can snap a picture, write a post and upload it for the parent to read. FACTORS is a method to define a product.

\subsection{Functions}
The contact book has a number of functions:
\begin{itemize}
	\item{Write a post of the day�s activites for the parents and kindergarten teachers to read.}
	\item{Both kindergarten teachers and parents should be able to edit and delete posts.}
	\item{Take, upload and edit pictures for the contact book. Furthermore it should be possible to attach a picture to the post of the day.}
	\item{The picture editor should include crop, resize and rotate.}
	\item{It should be possible to upload posts from both a mobile device and a PC.}
\end{itemize}

\subsection{Application domain}
The application domain includes the kindergarten teachers, parents and to some degree the autistic children:
\begin{itemize}
	\item{The kindergarten teacher should have access to the contact book of each child in an associated group. They should be able to upload/edit pictures and add them to the contact book.}
	\item{A parent should only have access to the contact book of their own child}
	\item{The child should have access to the contact book but without editing privileges.}
\end{itemize}

\subsection{Conditions}
There are several requirements for the contact book to work:
\begin{itemize}
	\item{There has to be synchronisation between mobile devices and the administration platform.}
	\item{It should be easily accessed by people with little IT-experience, as we have very little knowledge about the users, especially the parents.}
\end{itemize}

\subsection{Technology}
There are a couple of technological requirements for the contact book to work:
\begin{itemize}
	\item{The platform should be supported by both mobile devices as well as PCs.}
	\item{A database to store information, such as user credentials.}
  	\item{A specially designed framework to handle the contact book application.}
	\end{itemize}

\subsection{Objects}
The main objects related to the contact book:
\begin{itemize}
	\item{Mobile devices.}
	\item{The autistic children.}
	\item{Pictures.}
	\item{The parents and kindergarten teachers.}
	\item{``A Day'', as the contact book is most likely edited from day to day and does not matter e.g. from week to week.}
\end{itemize}

\subsection{Responsibility} 
%The main idea behind this contact book is to:
The contact book's functionality is to:
\begin{itemize}
	\item{Aid to the completion of writing and adding pictures to posts.}
	\item{Improve the communication between kindergarten teacher and parent as well as the communication between a parent and it's child.}
\end{itemize}

\section{System definition}
The contact book software should primarily be a tool for enhancing communication between parents and kindergarten teachers. It should enable and store daily communication with text and pictures between the parents and kindergarten teachers. This should be implemented on a platform, supported by both mobile devices and PCs.
\newpage