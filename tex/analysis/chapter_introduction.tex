\chapter{Introduction}

%The purpose of this project is to create a user-friendly web-application that makes the user able to create entries in a web-based contact book which should replace the current physical form of the contact book version and make it easier for the kindergarten teachers and guardians to communicate with each other. 

This project is created as an addendum to previous GIRAF projects, casting light on how additions can be made to the already implemented software.
The purpose of the project is to digitise a contact book and establish a foundation for further development within the GIRAF concept.
The digitised contact book is developed to replace the current physical form of a contact book, used in institutions like Birken.
Birken, which contributed to the project, is a kindergarten for children with need for special care and alternative learning environment. Birken is situated in Vodskov in the Northern Jutland.

Our contact from Birken, Kristine Niss-Henriksen, has been included in this project because of her knowledge about children with autism and experience with today's technology. She contributed with design ideas for the user interface of the system. Furthermore, she provided feedback in testing and on prototypes, which helped us create the final product.

In the beginning, the intention was to develop an administration tool for the GIRAF mobile devices, making it possible to manage and push application settings to registered devices. However, from the interviews with Kristine we learned that this functionality was not a pressing issue; adding the aforementioned contact book functionality had a higher priority.  

In the list below are definitions for the reader's convenience
\begin{description}
\item[Guardians:] Parents, or persons who have custody over the child
\item[Kindergarten teacher:] A person working with autistic children
\item[GIRAF:] While the groups of the multi-project of spring 2011 worked almost exclusively with Android-based applications, this project has expanded the application domain to the web. Hence, we define the following syntax for referring to various parts of the GIRAF concept: GIRAF-X-YY where X is a platform abbreviation (A for \emph{Android} and W for \emph{Web}) and YY is a year in two digits. Thus the multi-project from spring is referred to as GIRAF-A-11 and this project as GIRAF-W-11.
\item[GIRAFAdmin:] The primary product of this project. A web-based administration, including a contact book, is referred to as GIRAFAdmin.
\item[PC:] Personal Computer.
\item[Mobile device:] A tablet or smartphone, running Android 2.2 or later.
\item[GIRAF application administrator:] User interface allowing user to manage mobile applications and their settings.
\item[Proper names \& acronyms:] Are emphasised the first time they appear in the report; following appearances are not emphasised.
\end{description}


%Furthermore, this project will try to establish a foundation, such that the functionalities created in this project can be added to the current GIRAF system.
%Specifically, one institution contributed throughout the project,
%moreover how it would improve the working environment for kindergarten teachers working with autistic children.