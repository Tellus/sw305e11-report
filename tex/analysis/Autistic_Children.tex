\section{Autism}
Autism is defined as a developmental disorder, which affects the brain's ability to develop communication- and social skills. Autism often appears during the first three years of a child's life, and the disorder is often diagnosed during the following years.
The definition of autism is broad, which means that some autistic children have different symptoms than their peers, but to diagnose, the \emph{Autism Spectrum Disorder} which consists of three basic symptoms, is used. These three symptoms are listed below.
\begin{itemize}

  \item{Lack of communications skills.}
   \begin{itemize}
     \item{The child has slow or no development of language. The child may use gestures to communicate instead of words, and may have trouble with maintaining focus in a conversation or have trouble starting a conversation.}
   \end{itemize}
   
  \item{Lack of social skills.}
   \begin{itemize}
     \item{The child may have difficulty making friends. The child may be withdrawn or may not respond to eyecontact and smiles. The child may treat other children and/or adults as objects. The child may rather play alone than play with others.}
   \end{itemize}

  \item{Repetitive and/or compulsive behavior}
    \begin{itemize}
      \item{The child may have unusual distress if particular routines are changed or not being followed. The child may perform repeated body movements. \cite{autism}}
    \end{itemize}
  
\end{itemize}

The exact causes of autism are still unknown, but are supposedly a combination of different factors. Some suspected - though unproven - causes, are listed below:

\begin{itemize}
  
  \item{Diet}
  \item{Changes of the digestive tract}
  \item{Poison by mercury}
  \item{The body's lack of ability to utilize vitamins and minerals properly}
  \item{Sensitivity against vaccines\cite{autism}}
  
\end{itemize}

Even though the factors listed above are mostly physical, the autism disorder is supposedly linked to unusual biology and the chemistry of the brain.
It seems that genetics are an important factor. E.g. it is more likely for identical twins, than fraternal twins or siblings, to both be autistic\cite{autism}.