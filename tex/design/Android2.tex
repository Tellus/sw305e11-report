\section{GIRAF synchronization framework}
The original project proposal was about creating an administrative PC tool which should be able to synchronize data for a GIRAF tablet device. This could be one device at the time, or many devices through a centralized management system. In the beginning of the design process this general idea was used and developed upon. From this, specifics to a solution was made, including a detailed design process, additions to the existing framework and some proof-of-concept code. After the two first interviews with Birken, this part of the project was discontinued, because the contact book application was more useful to them (see appendix \vref{first_interview_birken}). Although discontinued, some of the code parts are fully functional; and it is still an important part of some of the initial project work that was made. Furthermore, the source code and designs made here, could be carried on as material for further development. 

The general idea included that the software should support user profile and application settings synchronization. Moreover, the existing GIRAF-android client has functionality to install and view available applications. In milestone 1.0 of the client software, this feature should be merged with the synchronization application, making it possible to control available applications through an Access Control List (ACL).

\subsection{Protocol, Client and server}
These 3 parts were designed in detail, and as mentioned earlier, some proof-of-concept code was developed. The protocol was designed to handle specific synchronization features, such as device identification, alignment of time differences - for handling clients located in different time zones. Furthermore details upon how to perform settings transfers, user profile synchronization and application transfer (also discussed how to handle updates).  

The client was designed to implement and extend the current database functionality from the \emph{sw6 admin} project. Design and example code was developed from the perspective that as few patches as possible should be made to the existing code made by the sw6 project groups.
The existing Android Client already supported installation of GIRAF applications through a simple GUI, this  was redesigned to be an automated process. Moreover, the client was designed to support update functionality as well. As a last feature, the client handles specific signal messages from the server (see \vref{ClientDesign}), as the server / client model in this case was designed only to send commands from client to server, and the server returns signal messages. In other words, the client does not accept any command messages, that could access database items on the client device. This approach was chosen to strengthen security.  

The server handles command messages received by the client, and communication with the backend MySQL server.
To summarize, a lot of work went into making the design, developing example code and discussing protocol functionality. To read more about the Android design process and milestone perspectives, the progress has been polished and included in appendix \vref{android} .
