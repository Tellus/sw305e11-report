%er under chapter future work som 1. section
\section{Further development}

\subsection{Android}
\label{androidReflection}
We did not make an application for the child such that he/she would be able to see the images and image text from the contact book, which was one of the important functionality of the contact book. However, the designs for the protocol, client and server have been documented. By implementing the features in protocol milestone 1.0 it would be much easier to develop an Android viewing application for the contact book, since it contains the basic functionality needed to synchronize and fetch data. In general; the next step would be to implement those features. 

\subsection{Synchronisation}
Late in the Android design process we discovered that there already exists a general synchronization class in Android called SyncAdapter. The SyncAdapter class basically has the same functionality as the general accounts you can add to the "My Accounts" page in Android\cite{AndDevel4}. It could be very interesting to investigate if this synchronization feature - with some extension - could synchronize our database items and applications as Google does with its own Gmail account. If possible, it would relieve us from making the synchronization service. Although at this point, we have not gone into depth with what can or cannot be done with SyncAdapter. 

Another possibility is to use the Oracle Mobile Server as backend - since it also features an Android synchronization client specifically designed to synchronize an SQlite database to an Oracle Database. Oracle also provides a Mobile Development Kit used for packaging, publishing and testing applications. Moreover it is free to use and deploy\cite{Oracle}.  

\subsubsection*{Milestone 1.0}
Milestone 1.0 contains milestone perspectives for the Android synchronization client, the synchronization protocol and server.  

\textbf{Client}\\
Synchronize content with the server\\
Pull updates on frequent basis\\

\textbf{Protocol}\\
Implement signal messages from client\\
Implement command messages from server\\

\textbf{Server}\\
Filter settings based on ident\\
Filter applications based on ident from client\\

\textbf{To do:}\\
Implement group settings.  


\subsection{myAccount and News}
\label{myAccount}
%Maybe it should say somthing like: MyAccount and the News homepage was designed but not complitily implented - Ren� rettet:lisbeth
MyAccount and the News homepage was designed but not implemented. The model for both my account and News was finished, but no controllers or views were completed.

Per design, My Account should  contain personal information and the users would be able associate themselves to a child the child to a device. Furthermore all the users connected children would be able to be made into groups by the user.

The News section would show announcements or global messages relevant either to all users or groups that the user is a member of. While regular users would only be able to read the messages, moderators of a group (such as kindergarten teachers) would be able to write new posts.

\subsection{Application administration}
We did not get to making an application administration system to aSchedule or DigiPECS. These administrations system would need the android synchronization such that the mobile devices will be updated. 

The aSchedule administration would be like a normal calendar that also uses PECS to illustrate the child's schedule for a day. 

DigiPECS administration would be a program where the user could use to make and edit PECS, and it should then have access to an public archive of PECS (evt. opensource where the user also can add PECS) and a private archive for private pictures and PECS. 

\subsubsection*{Personalize}
This section should describe the "Settings" application, in which the user can edit information about the child and add/remove eventual handicaps. (It should also not be called "Settings" as it is an insensitive name.


\subsection{Site Admin}
Administering applications, groups and children are only the outermost pieces of the full scope of site administration. Tools for super users such as user administration and managing global application access (e.g. auditing and authorising new controllers for recently-added Giraf-compliant applications) should be staples.

\subsection{Delevoper guide}
During development of the first architecture it became evident that expanding the system was quite difficult, even with comprehensive code documentation. Writing a developer guide that describes philosophies, conventions and inspirations would be the next step in opening up the system to new functionality.

\subsection{Remote installation}
At one point it was considered that applications could be installed remotely, that is without a user installing from the device, instead marking applications for installation through the web site and the device performing the install on its next synchronisation.