%er under chapter future work som 1. section
\section{Android}
%We did not make an application for the child such that he/she would be able to see the images and image text from the contact book, which was one of the important functionality of the contact book. However, the designs for the protocol, client and server have been documented. By implementing the features in protocol milestone 1.0 it would be much easier to develop an Android viewing application for the contact book, since it contains the basic functionality needed to synchronize and fetch data. In general; the next step would be to implement those features. 

\subsection{Synchronisation}
Late in the Android design process we discovered that there already exists a general synchronization class in Android called SyncAdapter. The SyncAdapter class basically has the same functionality as the general accounts you can add to the "My Accounts" page in Android\cite{AndDevel4}. It could be very interesting to investigate if this synchronization feature - with some extension - could synchronize our database items and applications as Google does with its own Gmail account. If possible, it would relieve us from making the synchronization service. Although at this point, we have not gone into depth with what can or cannot be done with SyncAdapter. 

Another possibility is to use the Oracle Mobile Server as backend - since it also features an Android synchronization client specifically designed to synchronize an SQlite database to an Oracle Database. Oracle also provides a Mobile Development Kit used for packaging, publishing and testing applications. Moreover it is free to use and deploy\cite{Oracle}.  

\subsubsection*{Milestone 1.0}
Milestone 1.0 contains milestone perspectives for the Android synchronization client, the synchronization protocol and server.  

\textbf{Client}\\
Synchronize content with the server\\
Pull updates on frequent basis\\

\textbf{Protocol}\\
Implement signal messages from client\\
Implement command messages from server\\

\textbf{Server}\\
Filter settings based on ident\\
Filter applications based on ident from client\\

\textbf{To do:}\\
Implement group settings.  

\label{androidReflection}
An Android version of the contact book application, tailored to the needs and capabilities of a child, was not completed. However, the designs for the protocol, client and server have been documented. Finalising the protocol implementation is the first stepping stone towards developing this application, given the need for some synchronising mechanism between device and site.

\section{Core site features}
Several features were planned and to some degree designed, but could not be completed before the project deadline. A subset of these features were envisioned as pseudo applications, accessible from the list of installed applications, while the remainders would be incorporated into the main site design.

\subsection*{My Account}
My Account was intended as a management area of the site, allowing users to manage their personal information as well as devices, groups and children. Only the Model component was implemented.

\subsection*{News}
News is a page available on the front page for the user to easily get updated on relevant news. The section could contain updates from the contact book, new applications available or new calendar entries. The Model was completed.

\subsection*{Personalise}
A pseudo application that would give access to modify a child's personal details.

\subsection*{GIRAFPlace}
GIRAFPlace was originally created as part of GIRAF-A-11 as a medium for publishing GIRAF-compliant applications. While the underlying database of that site and GIRAF-W-11 have been merged, their user interfaces and architectures are still independent. Merging the two into a single application would simplify the overall structure both for users and developers.

\subsection*{Site Administration}
Administering applications, groups and children are only the outermost pieces of the full scope of site administration. Tools for super users such as user administration and managing global application access (e.g. auditing and authorising new controllers for recently-added Giraf-compliant applications) should be considered necessary.

\subsection*{Gallery and PECS administration}
During the interviews with Birken, two specific products were highly desired. While one (the contact book) was designed an implemented, the other - a combined gallery and PECS constructor - was considered too complex for this project. As such, an open-source competitor to the proprietary BoardMaker application should be considered for future development.

%(and, to the interviewee, difficult to use)

\section{Administration of GIRAF-A-11 applications}
Referring to the earlier figure \vref{fig:classOversigt} (GIRAF Application Administrators), it was suggested that additional modules should be implemented in GIRAF-W-11 that allowed for relevant administrative options over the applications available for the Android. The synchronisation protocol as well as relevant sets of MVC components should be developed for this to be achieved.
% and these applications were not made in this project, but it were taken into account, that it should be implemented. The GIRAF applications DigiPECS and aSchedule would each need an administrator application so the user would be able to administer the application on the mobile device. However, for this to work the synchronization service has to be completed as was discussed in both chapter \vref{androidDesign} and subsection \ref{androidReflection}.
% GIRAF Application Administrators refers to the earlier figure 
Specifically the DigiPECS and aSchedule need administration modules on the same level as contact book.

%\subsection{DigiPECS application administrator}
%DigiPECS' administration application would be a program that the user could use to find, make and edit digital PECS more easily. Therefore it should have access to a public archive of PECS, which could be an open-source alternative to Boardmaker, where the user also could choose to save the DigiPECS they make. It should also have access to a private archive with private pictures and PECS, so the Personal Data Protection Act are complied with according to Danish law. 

%This idea originated from the GIRAF application DigiPecs and from the first interview with Kristine Niss-Henriksen about 51 minutes into the interview (see appendix \vref{first_interview_birken}).

%\subsection{aSchedule administrator}
%The aSchedule administration application would be much like a normal calendar, but it would also be using the DigiPECS to illustrate the child's schedule for a day. An entry in the calendar would as minimum contain an activity, description of the activity by using DigiPECS, a start time and possibly the duration of the activity.

\section{Developers guide}
During development of the first architecture it became evident that expanding the system was quite difficult, even with comprehensive code documentation. Writing a developer guide that describes philosophies, conventions and inspirations would be the next step in opening up the system to new functionality from third-party developers.

\section{Remote installation}
The current GIRAF client already contains functionality to fetch and install applications from the GIRAFPlace application server, this functionality is planned to be extended with an update feature, where the synchronization service pulls the server for updates making it into an automated process. This could be extended to include remote installation of applications on the device which Android currently features. Inspiration on how such feature could be implemented using the Android service class is accessible \cite{DevRemote}. 