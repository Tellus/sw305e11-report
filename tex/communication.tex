\section{Communication with an autistic child}
The parent and kindergarten teachers use Picture Exchange Communication System \textit{PECS} to communicate with an autistic child and this system has been successfully used for over 10 years in the USA. The communication tool consists of pictures or pictograms that represent something, the child is to do or something it wants to do. This could be a drawing of an apple and then the child knows that it is time to eat an apple. 

This form of communication is used within kindergartens for children with special needs e.g. children with autism. Mostly these pictograms are printed from a computer program called Boardmaker, which has many different pictograms that can be edited to make it easier for the child to understand. This could be when the child needs to put on a blue t-shirt, then the adult gives the child a pictogram of a blue t-shirt and the child understands. It can be confusing for the child if the actual t-shirt is one color and the one on the pictogram is another.

The PECS can also be used as a schedule for the children. In the kindergarten \textit{Birken}. Kindergarten teachers in Birken, use the pictogram as a daily schedule, where they put the pictograms on the wall so one row represents the activities of the day. Then the kindergarten teacher teaches the child to take the pictogram of the finished activity and put it away in a box. Later the pictogram can be put on next day's schedule, without confusing the child about whether the activity is finished or not.
%Later in the child's development the pictogram would be put next to the schedule, but the child would still understand that this activity is finished.
