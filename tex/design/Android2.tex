\section{GIRAF synchronization framework}

The original project proposal was about creating an administrative PC tool which should be able to synchronize data for a GIRAF mobile device. This could be one device at the time, or many devices through a centralized management system. In the beginning of the design process this general idea was used and developed upon. From this, specifics to a solution was made, including a detailed design process, additions to the existing framework and some proof-of-concept code. After the 2 first interviews with Birken, this part of the project was discontinued, because the contact book application was a more useful to them\vref{first_interview_birken}. Although it was discontinued, some of the code parts are fully functional; and it is still an important part of some of the initial project work that was made. Furthermore, the source code and designs made here, could be carried on as material for further development. 

The original project proposal was about creating an administrative PC tool which should be able to synchronize data for a GIRAF tablet device. This could be one device at the time, or many devices through a centralised management system. In the beginning of the design process this general idea was used and developed upon. From this, specifics to a solution was made, including a detailed design process, additions to the existing framework an some proof-of-concept code. After the two first interviews with Birken, this part of the project was discontinued, because the contact book application was a more useful to them (see \vref{first_interview_birken}). Although discontinued, some of the code parts are fully functional; and it is still an important part of some of the initial project work that was made. Furthermore, the source code and designs made here, could be carried on as material for further development. 

The general idea included that the software should support user profile and application settings synchronization. Moreover, the existing GIRAF-android client has functionality to install and view available applications. In milestone 1.0 of the client software, this feature should be merged with the synchronization application, making it possible to control available applications through an Access Control Lists (ACL).

