\section{First interview with Birken}
\label{first_interview_birken}

%klarg�r at den eneste kilde her er Kristine
Even before our input there was a clear desire to utilize IT to extend and improve communication by words.

Birken has a single child capable of working with actual words, otherwise the children most effectively communicate using Picture Exchange Communication System(PECS). However, the sounds, names and textual representations are very important in the discourse with guardians. Although a child may only signal a meaning with the image, they respond very strongly (and positively) to a reaction from the guardian if they respond with the word and possibly a complete sentence. While the child cannot speak the word, they can recognize it and the agreement in meaning is essential.

A flexible approach to imagery is very important. Images that most people would immediately understand with an exact meaning can be much more difficult for an autistic child to put into context. As a concrete example, Kristine mentions that a child she worked with was unable to comprehend their usual image for washing hands \todo{Get me an image of it.}, instead focusing intently on the head of the tap that to him resembled a power plug. These very particular needs play an important role when the guardians construct imagery via their current solution, Boardmaker, where they are currently forced to take a stock photo, manually (by pixel) delete unwanted parts of the image, and superimpose another to generate the desired result.

We thought that it would be helpful to know explicitly when a change is made to a known data set. Likewise, the administration module should not aim to create a remote chat feature with the children or changing a schedule without their knowing, but preferably asynchronous planning. What Kristine suggested at this point, was to have a default set of daily schedules that could be easily modified for daily use. She mentions that although they work with somewhat static weekly schedules (meal times, for example, or riding every Friday). She confirms that some children have sufficient capacity to memorize their entire daily schedule after seeing it at the start of the day - thus any changes need to be actively communicated to the child.

When asked about the weekly schedule, Kristine mentions that from one week to the other, the weeks are somewhat rigid. Particularly when external sources (personnel) are involved - this involves riding on Fridays and various health assistants on other days which would be difficult to change each week. Changes are allowed if deemed necessary, as well as teaching children the inevitable changes of even the most regular schedule.

A single day:
\begin{description}
	\item[7.30 to 8.50] The children arrive
	\item[9.00] Preparations for breakfast. Toilet visits, diaper changes and most importantly washing hands.
	\item Song.
	\item Variable time - either static weekly event or a choice of several activities, given to the child.
	\item Reading.
	\item Washing hands.
	\item Lunch.
	\item Various activities.
	\item Playground time.
	\item Washing hands.
	\item Fruit.
	\item Possibly new diapers, toilet visit.
	\item Taxi cab home.
\end{description}

As Kristine puts it, the absolute parts of a schedule are meal times (ALWAYS preceded by washing hands) and song right after breakfast.

(16:00) Upon suggesting that autism causes a very rigid mental model, Kristine confirms the almost mechanical precision of regular occurrences. This is however also dependent on taxi cab times.

(17:45) Kristine remarks that she has yet to meet a child with autism that knows the meaning of a clock. They use symbols, however, to signal milestones at intervals (for example, they will stick a clear arrow to a point on a clock face to denote where a hand needs to be before an activity ends).

On a peripheral question (regarding the individuality of grown children), the general approach used with the children Treatment and Education of Autistic and related Communication Handicapped Children (TEACCH) has been in use for at least 10 years.

The support of the children continues throughout their education, but Kristine emphasizes that she does not know any statistics on how well the children develop later on, remarking that in regard to education they are no different from children without autism.

(22:00) Based upon an earlier reference from our supervisor, Ulrik Nymann mentions that the primary method of communication between parents and supervisors happens through the physical contact book that the child brings back and forth every day. When first admitting a child the two parties very specifically, and in great detail, learn about the child and basic likes and dislikes. Direct contact is made when unexpected or somewhat acute events occur, either to discuss or request recommendations on working with the child.

(24:00) On the question about use of pedagogical terminology between parents and supervisors, Kristine explains that it would mostly be applied in situations where very exact expressions are necessary. Personally, she would use the exact terms and, if necessary, explain them. In other contexts, the usage of the exact terminology can seem alienating and create unnecessary complications, as has been the case with immigrants where basic Danish is still an issue.

(27:05) On using pedagogical terminology in a contact book application's user interface designed for parents Kristine deflects and specifies that the contact book is also used by the child themselves - most of them understand that it is an extension of themselves. Given daily reports and images of the child during an activity, the parents can use the book as a starting point when reflecting upon the day with their child.

(28:40) Suggesting digitizing the contact book, Kristine explains the currently lengthy process of placing images in the contact book. They use digital cameras, importing the images into a computer. Then they crop, resize and rotate the image as necessary (a process she describes as lengthy and hints at usability issues with current applications - in particular she remarks that if the one particular application she prefers to use is unavailable, %How can the software be unavailable. Is it only installed on some of the machines?
 then she cannot prepare the image), finally printing the modified image and gluing it to the book page.

(29:15) It is very important to Kristine and her co-workers to offer the images in order to facilitate communication between parents and children about the past day. Particularly focusing on the child's understanding of images; images within the contact book would aid them in reflecting on a situation or experience (while they may only have experienced riding a horse from within "their own head", seeing themselves on the horse from a third person's viewpoint gives greater detail).

(31:20) At this point the mobile device is discussed almost exclusively as a digitalization of the contact book. During idle time (in-between exercises, or when the kindergarten teachers need to write down the day's events) they will look in their book. 

(32:45) Discussing if the children could start using the device as an extension of themselves (constantly at hand with images available), we debate whether they should have it at hand always. Kristine sets up a simple rule: when determining what activity is to follow, the schedule is used. During an activity, it should be put aside. 

(34:00) A free thought experiment, we discuss having a sand timer on the device or possibly lock the device during an activity. Such a lock should be immediately switchable but could also see beneficial usage when set for particular children (she mentions some children are capable of understanding the boundaries of staying within a given activity without delving into other things).

(36:30) Alternative methods of communication. Most (if not all) of Birken's children need a structure in discourse. Given autistic childrens' intense focus on the use of images, some require that images are not accompanied with sounds.

(37:40) Asking to a combination of autism and blindness, Kristine has not encountered the issue (neither in theoretical or real contexts), but suggests the same approach to understanding as with other children is used. When teaching new meanings, children are given real versions (or scale models) of the activity or meaning being described (an actual apple or a small toilet), slowly moving to photo-realistic images of the item and finally to simpler drawings and imagery. Simply replacing vision with touch, the same approach is suggested for a blind child.

Kristine remarks that while the children have difficulty communicating, it may very well be that they have a very clear internal understanding of contexts and abstraction, but are simply unable to communicate it outwards. 

(40:45) There are no discrete steps applied for the children, but they do discuss a child's physical age with their estimated mental age. Kristine notes that they will never use specific terms to describe a child to the parents (cold, neutral terms, using facts instead of feelings), instead discussing higher level abstractions. 

(42:55) The children are registered in an electronic patient journal in a centralized system. Given the security criteria and very sensitive nature of the documents, I suggested not considering this a point of interest for the group. It is, however, completely open to the relevant parties (parents, guardians and assigned medical professionals).

(44:20) Kristine mentions their own vision. A tablet (a small, thin screen) that can be brought around, but also hung up, and taken home. Drawing inspiration from conventional kindergartens, Kristine mentions chat or forum applications that are offered to allow communicate between parents and supervisors.

(47:00) Kristine notes that there is a remarkable difference between the visions and needs of kindergarten teachers and parents.

(47:40) Open question: can Kristine imagine any other ways that mobile phones, tablets or PC's could ease their daily lives. She re-iterates the desire for a centralized (and most of all secure) application that allows for communication between parents and guardians. She suggests the easier and more direct communication makes for more information between the two parties. Mobile schedules are also interesting. Currently they do bring portable schedules in PECS form on field trips. However, they suffer from lack of available PECS that may be necessary, something that could be alleviated by a gallery always available online. She also suggests the flexibility of imagery, and options when creating a series of images. For example, some children need instructional images in a series (say, how to go to the toilet) to be enumerate either by numbers or letters, while others will be distracted by the same enumerations.

(50:00) Kristine confirms that the contact book/communication and flexibility of imagery is core to their visions.

(51:00) Kristine demonstrates BoardMaker and its features. Of particular note is that she does not know every feature that the application offers - it is unknown if the application is able to do offers some of the functions she requires. She demonstrates the creation of a new image. First delimiting a rectangular area for the image proper, she then searches its gallery for any images with a name matching a given string (food, for example). She finds an image of a stick figure holding a spoon to its mouth. A small blob of green mass is on the spoon. Deciding that the spoon and its contents would be disruptive to the hypothetical child in question. She continues to use an erase function to remove the spoon from the image, accidentally removing part of the stick characters face in the process. She explains that the process of removing unwanted items may take several attempts to avoid removing necessary elements. Moving on, she proceeds to find a fork in the gallery, and attempts to superimpose it on the stick figure. This takes several attempts as she cannot clearly remember the procedure to do so.

(1:09:00) Kristine describes a "social story" as a method of creating contextual meanings and discourse with a child.

(1:10:00) Kristine mentions that she may bring children with her to the computer so as to create images in cooperation with the child in order to maximise success rate of understandings.

(1:12:00) Kristine introduces 6 phases of TEACCH that describes learning curves, from phase 1, where the child is taught very simple progressive sequences. From phase 1, an image of something they like (preferably something they are already familiar with - a treat, for example) is shown. A guardian behind the child works their arms in order to pick up and present the image to a guardian in front of them. That guardian will take the image, say the associated word, and present a real version of the object (the treat). This process is repeated to teach the child the basic mechanics of communication in a context. Depending on the functional level of the child, phase 1 is repeated in new contexts (that rely on the same basic principles of communication) in order to emphasize the similarities. This continues up to phase 6 with increasing levels of independence and risk-reward situations (phase 1 will never result in a refusal of the child's request, while later phases will introduce this possible outcome).

%ulrik:" TEACCH (Treatment and Education of Autistic and related Communication Handicapped CHildren) "Established %in the early 1970s by Eric Schopler and colleagues, ..." Det er sikker brugt i Danmark i 10 �r eller p� Birken i %10 �r."