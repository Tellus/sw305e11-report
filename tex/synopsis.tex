This project is based on an Android system developed by four sixth semester groups in the spring semester of 2011 - it was called GIRAF. Based on the vision of a PC-based tool for administering GIRAF application and in cooperation with Birken (a kindergarten for children suffering from autism), this project expands the original concept into one a web-based framework for administration of GIRAF and tools focused on the communication between the various interest groups in play. Due to time constraints, only a part of the system was implemented, concentrating on the development of a digital contact book, to replace the physical books currently in use at Birken.

% This project is based on an Android system made in a bachelor multiproject in 2011. The developed system was called GIRAF. The initializing problem from the original project proposal, was to make a PC tool, used for administration of the applications developed for the GIRAF system. During the project the group has been cooperating with Birken, which is a institution for autistic children. After the first meeting with the representative Kristine Niss-Henriksen, the focus of the project was changed from making an administration tool, to making a digital version of a contact book; like the one used by Birken to communicate between the institution and the parents.

The report specifically describes the development of the central notion of a GIRAF administration system, singling out the contact book. The final system is implemented using web technologies (HTML, CSS, PHP, JS) in a Model-View-Controller architecture. Finally, the system's functionality and usability is lightly tested.

%This report describes the development of GIRAF administration system with main focus on the contact book. This report includes the usage of models, prototypes, the client-server architecture, and the implementation of the model-view-controller design. Furthermore to evaluate the system both unit testing and usability testing were used.