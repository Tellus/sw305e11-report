% under chapter reflection som 5. section
\section{Expanding the problem domain}
During this project we have based the product upon the needs of a institution that has children with autism, however, the GIRAF-A-11 could be used by both children and adults with different disabilities and therefore the GIRAF-W-11 could be used by other institutions' employees, that works with these individuals or family members to the person with the disability. The contact book application could be used in its current functionality in schools, but it could adapted to become more like a diary for the adults with a disability.     

The GIRAF concept could be expand further to support individuals having problems using a mobile device. An example of this would be the elderly population that is uncomfortable using a modern mobile device because of the extensive functionalities and it small buttons. To accommodate these users a safe environment with different levels of restrictions could be made such that administrator of the device(the GIRAF-W-11 user) and the user would be able to sort in which application the user wants to have.\todo{Er det her stadig inden for vores concept? -lisbeth. Ja det vil jeg mene - Ren�}

The GIRAF concept could also be used for children without any disability, where GIRAF Administration and the use of a contact book then also could enhance communications between parent and kindergarten or school teacher. Furthermore the GIRAF-W-11 then could include an application that forwards the child's homework to both the child and the parents, which also get solutions such that the parents are able to help the child with the homework. 

If the framework reaches a stable release (Android synchronization client, synchronization server and the possibility to manage groups) any organization or group in need of a closed application distribution and management form could use code to some extent - taking this a step further could be to extending the GIRAF platform to include and manage any type of application. 

    