\section*{Third interview with Birken}
\label{•}

0.00
Prototypen sættes op
Kristine bedes oprette en ny bruger

1.00
Prototypen virker ikke, så Johannes fortæller om projektet.

2.12
Skiftes til den bærbare, da IE ikke kan køre prototypen.

9.30
Kan ikke bruge 'Enter' til at submitte. Kristine forventede fokus i login-skærmen.

11.40
Kristine præsenteres en oversigt over børnene. 
Nogle bugs med menuen.

13.00
Kristine bedes lave et nyt indlæg i kontaktbogen. Hun forveklser boksene med hinanden.

14.00
Kristine bedes tiltøje et billede og vælger et billede fra computeren.

15.00
Billedet oploades succesfuldt 

16.00
Billedet er tilføjet med teksten, men viser også billeder fra et andet tidligere indlæg. 

16.40
Der diskuteres, om de vil have muligheden for at svare tilbage med billeder, hvilket Birken er positive overfor. 

17.30
Kristine forsøger at skrive et svar til indlægget.

18.30
Kristine synes svarfunktionen fungerer godt, med dato, afsender og besked.
Hun vil gerne prøve at lægge et billede ind og svare.

19.30
Vi fortæller om vores idéer om det samlede projekt/løsning, med en nyhedsside og generelle beskeder.

20.40
Kristine synes godt om funktionaliteten med uploadbillederne.
Der diskuteres om der skal være billedtekst eller ej.

21.20
Skal være tilgængeligt for barnet.

21.40
Diskutere, hvilke funktioner billedredigering skal være tilrådighed.
Kristine lægger mest vægt på beskæring.

22.30
Kristine spørger til en printfunktion. Hun vil gerne have en generel printfunktion til alle indlæg, som er tydelig og let tilgængelig.

23.30
Kristine nævner datalovgivningen, da billeder ikke må ligge mere end 3 måneder på computeren, hvilket betyder, at en printfunktion vil være nyttig, da barnet kan beholde billederne.

24.00
Diskussion af persondataloven, hvor der søges på internettet efter information.

28.30 
Søgen opgives, men diskussion fortsætter.

30.40
Problemstilling med at have serveren til at stå i England, hvor lovgivningen kan være forskellig.

33.00
Der spørges indtil Birkens budget, men Kristine ved intet, da det er lederens job. Der diskuteres økonomi for at have egne servere stående. 
Kristine informeres om forskellige muligheder inden for dette område.

34.40
Der diskuteres aflastningstilbud og vores skelning mellem pædagoger og værger

37.00
Kristine åbner op for muligheden for at alle børn kan bruge systemet.

38.10
Der diskuteres om resten af projektperioden.

40.30
Kristine nævner de positive ting ved designet
 - Nemt at logge ind
 - Nemt at klikke på børnene
 - Nemt at se og læse nye beskeder
 - Nemt at uploade beskeder
 - Nemt for ikke IT-erfarne
 - Nemt at forstå, hvilke funktioner de forskellige knapper har
 
41.15
Kristine sætter fokus på en printfunktion til kontaktbogen. 

42.00
Kristine forventer ikke mere af det visuelle design end det, der er lavet.

44.00
Fortæller Kristine om princippet bag en usability test. Sætter også fokus på problemet ved en ældre browser.
Opsummere interview/testen indtil videre.

46.30
Mere forklaring af design, herunder princippet af menuen med barn/applikationer

48.00
Diskussion af ordet ``Indstillinger``, som kunne være ufølsomt, hvor Kristine er enig. Hun foreslår ``Kontaktoplysinger``

49.30
Kristine foreslår en form for galleri, hvor man kan samle samtliger billeder til et fotoalbum el. for børnene. Muligheden for at åbne GIRAFAdmin for børnene kommer på tale.

51.50
Mystisk bug med menuen opdages af Kristine.

53.30
Kristine foretrækker at ``new`` skal forsvinde, når man har læst indlægget, hvor hun derefter sammenligner med e-mail og Outlook. Hun nævner også muligheden for at kunne markere et indlæg, hvis man skal vende tilbage til dette. 

54.30
Der diskuteres, hvorvidt det skal være synligt for personer at se, om indlægget har været læst af andre. Kristine foretrækker at have en distinktion mellem disse. Dog skal alle mails læses af alle pedagoger, og konklusionen bliver, at ikke have denne sammenhæng mellem indlægget og brugerne.

56.30
Funktionen, som gør det muligt at se, om der er kommet svar på et indlæg diskuteres, og Kristine er positiv overfor ideen. 

58.40
``Nyt indlæg`` layout diskuteres igen, hvor Kristine kommer med forbedringer.

61.40
Der spørges ind til ideen om at have et profilbillede, hvor Kristine virker positiv overfor ideen. 

62.40
Der diskuteres forskellige former for grupperinger af børnene, som kunne være praktisk, da børnehaven i Vodskov har 2 grupper af børn. Kristine er også positiv overfor ideen.

65.30
Kristine får adgang til systemet, hvor hun kan vise hendes medarbejdere hjemmesiden.

68.30
Kristine refererer til et sms-system, som bruges af en medarbejder.
