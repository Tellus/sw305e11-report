\chapter{Conclusion}
In this report we primarily explain the development process of a contact book, created for employees of kindergartens working with children who has special needs and for the children's parents as well. The process contains interviews from where we received user feedback from a representative for these employees.

The contact book was implemented with all the earliest defined functionality - however, certain advanced features defined later were left unfinished. The contact book was implemented as a module of a web based application, the purpose of which is administering GIRAF applications and their settings. Because of limited development time and resources, the group focused primarily on the functionality of the product and less on security. Despite the group�s focus, the aspect of security is important, because the application handles sensitive personal information, on both the client- and server-side. A few basic security measures were implemented; some basic security holes were closed during development, however, several key types were left in the concept stage.

The web system, in it�s current state, is ready for further first- and third-party development through a documented framework that was designed and partially tested. While developing the system, a protocol for accessing information on a GIRAF mobile device was also developed, but not implemented.

To summarize, we have a working version of the contact book which meets the requirements of our usability feedback and testing. We have designed a framework to help future developers integrate their software into the existing system and a protocol which makes communication with the GIRAF mobile device easier. After analysis within related domains, developing and testing, we can conclude that our solution meets its basic requirements; basic functionality of the contact book is available and the web administration framework is partly implemented.