\section{Development languages}

\subsection*{HTML and CSS}
\label{program_tools_html_css}
\emph{HTML} (HyperText Markup Language). HTML consists of elements that define what type of data the browser should display, primarily defining data and structure of a web page. Layout and general styling is instead defined with \emph{CSS} (Cascading Style Sheets), which is designed primarily to separate the website contents from the presentation.
The separation simplifies overall structure and readability for developers and in broader contexts allows selective loading of resources by automatic agents\cite{html}\cite{css}

\subsection*{PHP}
\emph{PHP}, or Hypertext Preprocessor, is a server-side scripting language. As the name inclines, it is preprocessed on the website server, and its output is sent to the client. Typically, PHP is used to generate output displayed in HTML. While the language can run recursive functions, generate code and transport data, it is static in view, all the work is handled by the server hosting the website, which then only transmits the output of executed scripts. To develop dynamic web content, JavaScript or equivalent \emph{client-side} scripting languages are applied.\cite{php}

\subsection*{JavaScript}
\emph{JavaScript}, not to be confused with \emph{Java}, is a scripting language that allows for the dynamic alteration of the \emph{DOM} (Document Object Model, a convention for representing objects in XML such as HTML). That is possible through client-side execution of web related scripts. The browser gets instructions from the website about how to execute the proposed content, where it is the user's choice of whether to run these instructions.
Since most browsers support dynamic web content through JavaScript execution, many developers run scripts on their websites to increase the visual aspect of the website and to make dynamic checking of forms and spelling for example.\cite{javascript} The dynamic impression of the site is further improved by \emph{AJAX} (Asynchronous JavaScript And XML) techniques, allowing the code to perform background operations (for example retrieving new data from the server) without impeding the display or reactions of the loaded page.
%\subparagraph*{jQuery}
\subsubsection*{jQuery}
\emph{jQuery} is a JavaScript library designed to simplify and ease client-side scripting on web pages. Among the headlining features are easier DOM traversal and manipulation, event handling and AJAX.

\section{Other concepts}

\subsection*{JSON}
JavaScript Object Notation (\emph{JSON}) is a human-readable notation typically used for the serialization of data across network connections. It is only semantically connected to JavaScript, and parsers for it exist in several programming languages - most importantly, JavaScript and PHP have encoding and decoding functionality for JSON, enabling platform-independent transfer of objects between the PHP-based server-side scripting and the JavaScript-based client-side scripting.

\subsection*{MySQL}
\emph{MySQL} is the database (formally relation database management system, or RDMS) supported by GIRAFAdmin. Although many alternatives exist (Firebird, OracleDB, PostgreSQL and SQLite to name a few) MySQL was chosen through a combination of language support (PHP has built-in support for MySQL), price (unlike some database systems like OracleDB, MySQL offers both free open-source and proprietary paid licenses) and adoption rate (many surveys place MySQL ahead of open-source competitors, and Oracle - owner of both OracleDB and MySQL - has the largest market share by revenue) \cite{mysqlpenetration}\cite{mysqlmarketshare}. Most RDMS support SQL (Structured Query Language), although all differ from the SQL standard to a certain degree. The current implementation of GIRAFAdmin only uses basic SQL statements considered globally available, and the explicit use of MySQL is placed in one location (the sql\_helper class) - extending the support to a new SQL-compliant RDMS should merely be a question of modifying that class to fit.

\subsubsection*{CRUD}
\label{CRUD}
CRUD, or \emph{C}reate, \emph{R}ead (sometimes Retrieve), \emph{U}pdate and \emph{D}elete (sometimes Destroy) is an acronym for the four basic operations of persistent storage (for example in databases). Using the acronym when describing an object simply specifies by shorthand whether that object features these operations.