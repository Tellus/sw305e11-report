\newpage
\section{UI functions}

In this section a list of functions, regarding the visual part of the system will be laid out. This is done to give a better understanding of how the visual part of the prototype was to handle requests and updates, from both the user and the server.
\\
\\
Function types:
\begin{itemize}
\item{Update: Functions of this type are activated by an event in the problem area. Result: state-change in the model.}

\item{Signaling: Functions of this type are activated by state-change in the model, resulting in reaction in surroundings. Reaction can be a presentation for the users involved in the scope or direct intervention in the problem area.}

\item{Readings: Functions of this type are activated by the need of information in a user's tasks. Result: system displays relevant parts of the model.}

\item{Calculation: Functions of this type are activated by the need of information in a user's tasks, consisting of a calculation, which involves information from the user and the model. Result: Displaying the result of the calculation.}

\end{itemize}

The simple, medium, complex tags indicate how simple or complex the functions are to implement according to the development group.

\begin{table}[!ht]
\centering
\begin{tabular}{ l  c  c }

Login Screen: &  & \\ \hline
\textit{Function name} & \textit{Function complexity} & \textit{Function type} \\ \hline
login & Simple & Update \\ \hline
forgotPassword & Simple & Update \\ \hline
register & Medium & Update \\ \hline
\end{tabular}
\caption{Login screen functions}
\label{tbl:loginscreen}
\end{table}

\begin{table}[!ht]
\centering
\begin{tabular}{ l  c  c }

Newsfeed: & & \\ \hline
\textit{Function name} & \textit{Function complexity} & \textit{Function type} \\ \hline
getNews & Simple & Readings \\ \hline
addNewsPost & Medium & Update \\ \hline
canUserPost & Simple & Signaling \\ \hline

\end{tabular}
\caption{News functions}
\label{tbl:newsfeed}
\end{table}

\begin{table}[!ht]
\centering
\begin{tabular}{ l  c  c }

Main menu: & & \\ \hline
\textit{Function name} & \textit{Function complexity} & \textit{Function type} \\ \hline
signOut & Simple & Update \\ \hline
getMyAccount & Simple & Readings \\ \hline
getGroup & Medium & Update \\ \hline
getGroups & Medium & Readings \\ \hline
getChildList & Simple & Signaling \\ \hline
getAppList & Simple & Signaling \\ \hline
setView & Complex & Readings \\ \hline

\end{tabular}
\caption{Main menu functions}
\label{tbl:mainmenu}
\end{table}

\begin{table}[!ht]
\centering
\begin{tabular}{ l  c  c }

Contact book: & & \\ \hline
\textit{Function name} & \textit{Function complexity} & \textit{Function type} \\ \hline
getMessage & Medium & Readings \\ \hline
getMessageList & Medium & Readings \\ \hline
addNewMessage & Medium & Update \\ \hline
getCurrentUser & Simple & Readings \\ \hline
uploadImage & Medium & Update \\ \hline
addNewReply & Medium & Update \\ \hline

\end{tabular}
\caption{Contact book functions}
\label{tbl:contactbook}
\end{table}

