GIRAF Application Administrators refers to the earlier figure \vref{fig:classOversigt}, and these applications were not made in this project, but it were taken into account, that it should be implemented. The GIRAF applications DigiPECS and aSchedule would each need an administrator application so the user would be able to administer the application on the mobile device. However, for this to work the synchronization service has to be completed as was discussed in both chapter \vref{androidDesign} and subsection \ref{androidReflection}.
   
\subsection{DigiPECS application administrator}
DigiPECS' administration application would be a program that the user could use to find, make and edit digital PECS more easily. Therefore it should have access to a public archive of PECS, which could be an open-source alternative to Boardmaker, where the user also could choose to save the DigiPECS they make. It should also have access to a private archive with private pictures and PECS, so the Personal Data Protection Act are complied with according to Danish law. 

This idea originated from the GIRAF application DigiPecs and from the first interview with Kristine Niss-Henriksen about 51 minutes into the interview (see appendix \vref{first_interview_birken}).

\subsection{aSchedule administrator}
The aSchedule administration application would be much like a normal calendar, but it would also be using the DigiPECS to illustrate the child's schedule for a day. An entry in the calendar would as minimum contain an activity, description of the activity by using DigiPECS, a start time and possibly the duration of the activity.