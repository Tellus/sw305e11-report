\chapter{System definition}
In this chapter a FACTORS analysis and a system definition (that will list the demands for the digitized contact book), is outlined.

\section{FACTORS (previously BATOFF)}
This is a FACTORS analysis of the online journal, a communication tool for parents and kindergarten teachers. The journal should be available from the tablet so that a kindergarten teacher can snap a picture, write a post and upload it for the parent to read. FACTORS is a tool to define a product.

\subsection{Functions}
The journal itself has a number of functions:
\begin{itemize}
	\item{Write a post of the day for the parents and kindergarten teachers to read.}
	\item{Both kindergarten teachers and parents should be able to edit and delete posts.}
	\item{Take, upload and edit pictures for the journal. Furthermore it should be possible to attach a picture to the post of the day.}
	\item{The picture editor should include crop, resize and rotate.}
	\item{It should be possible to upload posts from both the mobile and the stationary device.}
\end{itemize}

\subsection{Application domain}
The scope for the journal concerns the kindergarten teachers and parents and to some degree the autistic children. Finally, there will be some administrators maintaining the system:
\begin{itemize}
	\item{The kindergarten teacher is a main user of the journal as he or she mostly fills out the journal. The kindergarten teacher should have access to the journal of each child in an associated group and be able upload and edit pictures to add to the journal.}
	\item{The parent should only be able to read the journal of their own child and be able edit/add pictures.}
	\item{The child should have access to the journal but without editing privileges.}
\end{itemize}

\subsection{Conditions}
There are several requirements for the journal to work:
\begin{itemize}
	\item{There has to be a connection between the mobile devices and the platform.}
	\item{It should be easily accessed by people with little IT-experience, as we have very little knowledge about the users, especially the parents.}
\end{itemize}

\subsection{Technology}
There are a couple of technological requirements for the journal to work:
\begin{itemize}
	\item{The platform should be supported by both mobile devices as well as stationary devices.}
	\item{A database to store information, such as user credentials.}
  	\item{A specially designed framework to handle the journal application.}
	\end{itemize}

\subsection{Objects}
The main objects related to the journal:
\begin{itemize}
	\item{Mobile devices such as tablets or smartphones.}
	\item{The autistic children as they can read the journal.}
	\item{Pictures taken for the journal.}
	\item{The parents and kindergarten teachers as the journal serves as a communication between the two.}
	\item{A day, as the journal is most likely edited from day to day and does not matter from i.e. week to week.}
\end{itemize}

\subsection{Responsibility} 
The main idea behind this journal is to:
\begin{itemize}
	\item{Aid to the completion of writing and adding pictures to posts.}
	\item{Give an administrative tool to manage the journal.}
	\item{Improve the communication between kindergarten teacher and parent aswell as the communication between a parent and it's child.}
\end{itemize}

\section{System definition}
The contact book software should primarily be a tool for enhancing communication between parents and kindergarten teachers, and secondly a helping tool for the child to better communicate with others. It should enable and store daily communication with text and pictures between the parents and kindergarten teachers. This should be done by a platform which is supported both on mobile devices and stationary devices.
\newpage